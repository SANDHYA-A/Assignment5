\documentclass[journal,12pt,twocolumn]{IEEEtran}
\IEEEoverridecommandlockouts
\usepackage{cite}
\usepackage{amsmath,amssymb,amsfonts,bm}
\usepackage{mathtools}
\usepackage{tkz-euclide} 
\usepackage{tikz}
\usetikzlibrary{calc,math}
 \usepackage{caption}
\usepackage{listings}
\usepackage{gensymb}
\let\vec\mathbf
\numberwithin{equation}{subsection}

\newcommand{\myvec}[1]{\ensuremath{\begin{pmatrix}#1\end{pmatrix}}}
\newcommand{\norm}[1]{\left\lVert#1\right\rVert}
\newcommand{\mydet}[1]{\ensuremath{\begin{vmatrix}#1\end{vmatrix}}}

\renewcommand\thesection{\arabic{section}}
\renewcommand\thesubsection{\thesection.\arabic{subsection}}
\renewcommand\thesubsubsection{\thesubsection.\arabic{subsubsection}}

\renewcommand\thesectiondis{\arabic{section}}
\renewcommand\thesubsectiondis{\thesectiondis.\arabic{subsection}}
\renewcommand\thesubsubsectiondis{\thesubsectiondis.\arabic{subsubsection}}
%\renewcommand{\theequation}{\theenumi}
%\numberwithin{equation}{enumi}

\providecommand{\mbf}{\mathbf}
\providecommand{\pr}[1]{\ensuremath{\Pr\left(#1\right)}}
\providecommand{\qfunc}[1]{\ensuremath{Q\left(#1\right)}}
\providecommand{\sbrak}[1]{\ensuremath{{}\left[#1\right]}}
\providecommand{\lsbrak}[1]{\ensuremath{{}\left[#1\right.}}
\providecommand{\rsbrak}[1]{\ensuremath{{}\left.#1\right]}}
\providecommand{\brak}[1]{\ensuremath{\left(#1\right)}}
\providecommand{\lbrak}[1]{\ensuremath{\left(#1\right.}}
\providecommand{\rbrak}[1]{\ensuremath{\left.#1\right)}}
\providecommand{\cbrak}[1]{\ensuremath{\left\{#1\right\}}}
\providecommand{\lcbrak}[1]{\ensuremath{\left\{#1\right.}}
\providecommand{\rcbrak}[1]{\ensuremath{\left.#1\right\}}}

\lstset{
frame=single, 
breaklines=true,
columns=fullflexible
}

\begin{document}

\title{Matrix Theory EE5609 - supporting Assignment 5\\
}

\author{\IEEEauthorblockN{Sandhya Addetla}\\
\IEEEauthorblockA{PhD Artificial Inteligence Department} \\
AI20RESCH14001\\
 }

\maketitle
\begin{abstract}
Relation between hyperbola and conjugate hyperbola in terms of $\vec{V} , \vec{u}$ and $f$. 
\end{abstract}
\section{Solution}
The general equation of second degree is given by
\begin{align}
ax^2+2bxy+cy^2+2dx+2ey+f=0\label{2.1}
\end{align}
and can be expressed as
\begin{align}
\vec{x}^T\vec{V}\vec{x}+2\vec{u}^T\vec{x}+f=0 \label{2.2}
\end{align}
where
\begin{align}
\vec{V} &= \vec{V}^T = \myvec{a & b \\ b & c}
\\
\vec{u} &= \myvec{d \\ e}
\end{align}   
Equation \ref{2.2} refers equation of a hyperbola if $|V| < 0$.\\
Equation of a hyperbola and the combined equation of the Asymptotes differ only in the constant term.
\begin{align}
 ax^2+2bxy+cy^2+2dx+2ey+(f + K)=0   \label{2.36}
\end{align}
The above equation \ref{2.36} represents equation of asymtotes for equation \ref{2.2} and can be expressed in the form 
\begin{align}
\vec{x}^T\vec{V}\vec{x}+2\vec{u}^T\vec{x}+f_1&=0 \label{2.37}
\intertext{Where } f_1 = f + K
 \end{align}
For a pair of straight lines, 
\begin{align}
\Delta=\begin{array}{|cc|}\vec{V} & \vec{u} \\ \vec{u}^T& f_1
\end{array} &= 0\\
\begin{array}{|cc|}\vec{V} & \vec{u} \\ \vec{u}^T& f+K
\end{array}& = 0\\
\begin{array}{|ccc|} a & b & d\\ b & c & e \\ d & e & f+K
\end{array} &= 0
\end{align}
expanding this determinant, we get,
\begin{multline}
ac(f+K) + 2bde -a e^2 -cd^2 \\-(f+K)b^2 =0
\end{multline}
\begin{align}
K = - \frac{(acf + 2bde -ae^2 -cd^2 -b^2f)}{ac-b^2}\\
K= - \frac{\begin{array}{|cc|}\vec{V} & \vec{u} \\ \vec{u}^T& f\\
\end{array}}{\begin{array}{|c|}{\vec{V}}\end{array}}
\end{align}
Let equation of the conjugate hyperbola be as given below
\begin{align}
ax^2+2bxy+cy^2+2dx+2ey+f_2=0\label{2.14}
\end{align}
 and it can be expressed in the form 
\begin{align}
\vec{x}^T\vec{V}\vec{x}+2\vec{u}^T\vec{x}+f_2&=0 \label{2.15}
\end{align}
 The Equation of Conjugate hyperbola is also given by:\\
\\
2(Equation of Asymptotes)- Equation of hyperbola.\\
\begin{multline}
2 \times (\vec{x}^T\vec{V}\vec{x}+2\vec{u}^T\vec{x}+f_1)\\ -(\vec{x}^T\vec{V}\vec{x}+2\vec{u}^T\vec{x}+f)=0 
\end{multline}
\begin{align}
\vec{x}^T\vec{V}\vec{x}+2\vec{u}^T\vec{x}+(2f_1 -f)= 0\label{2.16}\\
\vec{x}^T\vec{V}\vec{x}+2\vec{u}^T\vec{x}+(f + 2K)= 0\label{2.17}
\end{align}
comparing equations  \ref{2.15} and \ref{2.17}, we get
\begin{align}
f_2 = f+ 2K\\
f_2 = f - 2 \frac{\begin{array}{|cc|}\vec{V} & \vec{u} \\ \vec{u}^T& f\\
\end{array}}{\begin{array}{|c|}{\vec{V}}\end{array}}
\end{align}
This shows the  relation between hyperbola and conjugate hyperbola in terms of $\vec{V} , \vec{u}$ and $f$. 

\end{document}